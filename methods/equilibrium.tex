\subsection{Equilibrium model}
\begin{frame}{Equilibrium model}
\label{sec:equilibrium_model}

\begin{columns}
\begin{column}{0.5\textwidth}

LLC works for TF\\
extend to PK with second node attribute

% The LLC model~(\autoref{sec:eberhardt}) attempts to describe causal effects from node value to another through unknown edges. The natural application to gene regulation would be to have each node represent both a gene and its protein product, node values $\boldsymbol{x}$ would represent gene expression levels, and edges would represent protein regulation. The node values can be absolute or relative protein concentrations, in our case log fold-change mRNA measurements~(\autoref{sec:yeast_data}).

% The direct causal effects from one protein to the expression of another makes sense when describing transcription factors that directly regulate gene expression, but is unsuitable when including protein kinases and other protein regulators, since their regulation are on another protein attribute that concentration, for instance the protein's level of phosphorylation. Their influence on the observed gene expression levels is only indirect, which would make them latent variables in LLC, where their effects will have to be described with $\boldsymbol{e}$ as linear effects.
% LLC can be extended to better model indirect regulation effect. Since many proteins can be assumed to be either transcription factors, kinases or phosphatases, it is possible to separate them in the model.

% In order to capture this indirectness, LLC is extended here by adding a second node attribute: "activity". Activity of a protein is meant to describe its state of phosphorylation but since it is unknown if a given protein is in its active form when phosphorylated or dephosphorylated, we generalize the concept of phosphorylation as protein activity.

% A graph in this model is therefore fully defined by a set of edges and nodes, where each edge are directed, so has source and target node attribute, has a signed edge value, and is either categorized as a transcription or activity regulating edge. Each node has three attributes: observed node value indicating relative concentration, unobserved activity attribute, and is category as either TF or PK. The "node value" refers to the observed node value if not explicitly referring to any of the three attributes. We use the simple assumption that a protein cannot be both a transcription regulator and an protein activity regulator, which means that edge types are implicitly known from the type of the edge source.

% The simplest case of regulation between node values is linear effects from node value to node value which can be described as by Eberhardt~et~al. The transcription factors will then have a linear regulation for directly observed node values, while the kinases have linear regulation on the activity attribute of nodes~(\autoref{eq:basic_eberhardt_extension}).


\begin{subequations}
\label{eq:basic_eberhardt_extension}
\begin{align}
x_i(t) &= \sum_{j \in T} w_{ij} a_j(t-1) + e_i^{(x)}
\\
y_i(t) &= \sum_{j \in P} w_{ij} a_j(t-1) + e_i^{(y)}
\\
\boldsymbol{x}(t) &= W I_T \boldsymbol{a}(t-1) + \boldsymbol{e}_x
\\
\boldsymbol{y}(t) &= W I_P \boldsymbol{a}(t-1) + \boldsymbol{e}_y
\end{align}
\end{subequations}
Node attributes:
\begin{conditions}
\boldsymbol{x}(t) & observed in data \\
\boldsymbol{y}(t) & hidden (PK regulated activity) \\
\boldsymbol{a}(t) & effective
\end{conditions}
% Here, $x_i(t)$ is the $i$-th directly observed node value, $y_i(t)$ the activity node attribute, each at discrete timestep $t$. $w_{ij}$ is the edge weight from protein $j$ to $i$, and $a_j(t-1)$ the effective amount of the $j$-th protein at time $t-1$. $e_i^{(x)}$ and $e_i^{(y)}$ are the error terms capturing latent, overlooked variables, and noise. The set $T$ and $P$ are the node indexes for transcription factors and protein kinases, so in matrix/vector notation we can combine all edges in a matrix $W$ where $I_T$ and $I_P$ are diagonal matrices where $I_T$ has 1s in the diagonal corresponding to indexes for TF nodes in $\boldsymbol{x}$, and similarly for $I_P$ in regard to PK node indexes. We sort the proteins with TFs first, followed by PKs~(\autoref{fig:W}).
% The matrix $W$ has zeros in its diagonal for the same reason as why this is enforced for $B$ in LLC~(\autoref{sec:eberhardt}).
\end{column}

\begin{column}{0.5\textwidth}
\begin{figure}[ht]
    \centering
    \includegraphics[width=\textwidth]{methods/fig/W.pdf}
    \caption{\textbf{\textit{W}.} Column = edge source, row = target. \\ \textcolor{darkgray}{Dark gray} = 0. Dimensions = $N \times (N_T + N_P)$. }
    \label{fig:W}
\end{figure}
\end{column}
\end{columns}
\end{frame}

\begin{frame}{Equilibrium model}
% Since we are only observing~$\boldsymbol{x}$ we wish to have all other terms for hidden node values disappear which is possible since the model is linear~(\autoref{eq:first_equilibrium_final}).

% If we describe each of the terms as log fold-change versions of absolute measurements for knockout and wildtype experimental conditions, we can get the following~(\autoref{eq:second_equilibrium_def}).
$(t)$ omitted:
\begin{subequations}
\label{eq:second_equilibrium_def}
\begin{equation}
a_i^{(\text{ko})} = y_i^{(\text{ko})} \cdot x_i^{(\text{ko})}
\quad,\quad
a_i^{(\text{wt})} = y_i^{(\text{wt})} \cdot x_i^{(\text{wt})}
\end{equation}\begin{equation}
x_i = \log \frac{x_i^{(\text{ko})}}{x_i^{(\text{wt})}}
\quad,\quad
y_i = \log \frac{y_i^{(\text{ko})}}{y_i^{(\text{wt})}}
\quad,\quad
a_i = \log \frac{a_i^{(\text{ko})}}{a_i^{(\text{wt})}}
= \log \frac{y_i^{(\text{ko})} x_i^{(\text{ko})}}{y_i^{(\text{wt})} x_i^{(\text{wt})}}
= y_i + x_i    
\end{equation}\begin{equation}
\boldsymbol{a}(t) = \boldsymbol{y}(t) + \boldsymbol{x}(t)
\end{equation}
\end{subequations}

% Here $y_i^{(\text{ko})}$ and $y_i^{(\text{wt})}$ are absolute node values, which are typically unmeasured, describing the effective fraction of each protein.

% Owing to the simplicity of a linear model we can cancel out hidden node values as before~(\autoref{eq:second_equilibrium_final}).
Can be shown to converge for $t\rightarrow\infty$
\begin{subequations}
\label{eq:second_equilibrium_final}
\begin{align}
\boldsymbol{a} &= 
    \left(WI_P\boldsymbol{a} + \boldsymbol{e}_y\right) + \boldsymbol{x}
\\
\left(I-WI_P\right) \boldsymbol{a} &=
    \boldsymbol{e}_y + \boldsymbol{x}
\\
\boldsymbol{a} &=
    \left(I-WI_P\right)^{-1} \left(\boldsymbol{e}_y + \boldsymbol{x}\right)
\\
\boldsymbol{x} &=
    WI_T \left(I-WI_P\right)^{-1} \left(\boldsymbol{e}_y + \boldsymbol{x}\right) + \boldsymbol{e}_x
\\
&= WI_T \left(I-WI_P\right)^{-1} \boldsymbol{x} + \boldsymbol{e}
\\
\label{eq:second_equilibrium_final.f}
&= B\boldsymbol{x} + \boldsymbol{e}
\enspace,\enspace B = WI_T \left(I-WI_P\right)^{-1}
\end{align}
\end{subequations}


% Again, the resulting description of observed node values $\boldsymbol{x}$ at equilibrium as a function of itself, can be written as by Eberhardt~et~al. with an extension to $B$. The extension of $B$ is identical to the one at \autoref{eq:first_equilibrium_final.e} with the exception of the 2.
% This model was chosen as the main method going forward based on better intuition in its design, as well as through tests showing little performance difference. 
\end{frame}


\begin{frame}{Equilibrium model - simulation}
\label{sec:prim}

Initial conditions (fig.~\ref{fig:KO_RNA_hist}):
\begin{subequations}
\begin{align}
\boldsymbol{x}_{\mathcal{U}_k} &= 0
\\
\boldsymbol{x}_{\mathcal{J}_k} &\sim \mathcal{N}(\mu=-4, \sigma=1)
\end{align}
\end{subequations}

Simple iteration of eq. \ref{eq:second_equilibrium_final.f} \\
% , where the parameters are chosen based on the observations in . $\boldsymbol{x}(t)$ is then calculated iteratively with \autoref{eq:second_equilibrium_final.f} until approximate convergence is reached, otherwise it will be stopped after 10,000 iterations. Approximate convergence is reached when it holds
Approx. convergence:
\begin{equation}
\label{eq:convergence}
\tfrac{1}{N} \sum_{i=1}^N |x_i(t)-x_i(t-1)| < \epsilon_{tol} \,,\,\epsilon_{tol}=10^{-7}
\end{equation}

\end{frame}

\begin{frame}{Equlibrium model - inference}
\label{sec:equilibrium_inference}

% The obvious way to implement the equations described in \autoref{eq:second_equilibrium_final} in order to infer graph edges will be by applying the algorithm of Eberhardt~et~al., which gives us $B$, and with \autoref{eq:second_equilibrium_final.f} solve for $W$. This cannot be done analytically but is implemented as a gradient descent method where the difference between $B$ and the right hand side of its definition in \autoref{eq:second_equilibrium_final.f} is minimized~(\autoref{eq:loss_B}).
% This approach is referred to as the $B$-method. If there is no covariance matrices for $\boldsymbol{x}_k$ available, $T_{\boldsymbol{x}_k}$ will have to be found using \autoref{eq:t_regress}, which only describes single intervention experiments.

Minimizing a loss function

% Another approach that allows for the inclusion of multiple interventions in an experiment is to minimize~$\boldsymbol{e}$ which is also an attempt at having the least amount of latent variables in the system.
% This approach is referred to as the $\boldsymbol{e}$-minimization method.
% It works by describing the systems of equations from the model in a symbolic math or neural network library in Python. The libraries help by constructing functions for gradients $\dv{\mathcal{L}}{w_{ij}}$, where $\mathcal{L}$ is the loss function and $w_{ij}$ a parameter in $W$. We then minimize the gradient using Adam gradient descent until perceived convergence~\cite{adam}.
% The graph will be sparse which is enforced through L1-regularization. The loss minimized for the $\boldsymbol{e}$-method and $B$-method are:

\begin{subequations}
\label{eq:loss}
\begin{align}
\mathcal{L}_B &=
\sum_{j=1}^N \sum_{i=1}^N
\left(b_{ij} - b_{ij}^{(\text{LLC})}\right)^2
+ \lambda_T \sum_{j \in \text{TF}} \sum_{i=1}^N |w_{ij}| + \lambda_\text{KP} \sum_{j \in \text{KP}} \sum_{i=1}^{N_\text{TF} + N_\text{KP}} |w_{ij}|
\label{eq:loss_B}
\\
\mathcal{L}_{\boldsymbol{e}} &=
\sum_{i=1}^N e_i^2 + \lambda_\text{TF} \sum_{j \in \text{TF}} \sum_{i=1}^N |w_{ij}| + \lambda_\text{KP} \sum_{j \in \text{KP}} \sum_{i=1}^{N_\text{TF} + N_\text{KP}} |w_{ij}|
\label{eq:loss_e}
\end{align}
\end{subequations}

% Here, $\lambda_T$ and $\lambda_P$ are regularization hyperparameters for $W I_T$ and $W I_P$, which means they are chosen to achieve a desired level of sparsity in each of those parts of the weight matrix $W$. They are not necessarily chosen as two different values. $b_{ij}^{(\text{LLC})}$ are values of $B_{\text{LLC}}$ found using LLC. $b_{ij}$ are elements of $B$ which is a function of the trainable parameters in $W$ as described in~\autoref{eq:second_equilibrium_final.f}.


\end{frame}
\subsection{Silent edges}
\begin{frame}{Silent edges}
\label{sec:unobservable}
\begin{columns}
\begin{column}{0.5\textwidth}

\only<1| handout:1>{
Edges silent in RNA logFC data simulated on arbitrary graph

% When inferring a protein-protein or protein-DNA interaction using RNA log fold-change values it is a basic assumption that the RNA values will be affected by the presence or absence of said interaction. In a graph model a TF edge will be observable if the target of the edge is a gene directly measured in the RNA log fold-change data.

% The protein kinases can have outgoing edges with TFs or other PKs as target~(\autoref{fig:unobservable}). If the target of their edge is a TF, then the PK edge will only be detectable if the same can be said for at least one edge from the TF to a measured gene. If the target of the PK edge is another PK, then in order to be detectable, the target will have to have at least one detectable outgoing edge. 


% For real data we will assume that enough genes are recorded that any direct protein-protein or protein-DNA interaction should have some level of effect on gene expression, however small, and that kinase regulation has an effect on gene expression. It is more relevant to consider undetectability for graphs designed artificially from more-or-less random adjacency matrices where it can occur that edges lead to nodes having no outgoing edges themselves.

% A nonsilent node is a node with at least one nonsilent outgoing edge. From a graph with known adjacency matrix $W$ we calculate which nodes are nonsilent with

\begin{equation}
\label{eq:detectable}
\boldsymbol{\omega} =
\sign \sum_{k=0}^K {|W|^\trans}^k
I_T |W|^\trans \boldsymbol{1} 
\end{equation}
\begin{conditions}
|W| & element-wise absolute of $W$ \\
\boldsymbol{1} & vector of 1s
\end{conditions}
% Here $\boldsymbol{\omega}$ is a vector where entry $\omega_i$ will be 1 if node $i$ is detectable and 0 otherwise. The superscript $\trans$ refers to transposing. $W$ has to be square, which it is in all cases where this equation has been applied. $\sign$ is the sign function, $|W|$ is the absolute of $W$, $\boldsymbol{1}$ is a vector of 1s with length $N_T + N_P$. $K$ is the length of the longest cascade of protein kinases, which is found through iterative calculation of $\boldsymbol{\omega}(K)$ as the smallest $K$ for which it holds that $\boldsymbol{\omega}(K) = \boldsymbol{\omega}(K + 1)$.


% \autoref{eq:detectable} can be read as starting with finding all TFs with at least one outgoing edges~($W^\trans \boldsymbol{1} I_T$), and iteratively follow all kinase cascades backwards for each iteration $k$.
}

\only<2| handout:2>{

% The simple network in~\autoref{fig:unobservable} can be used as an example, where we assign some random positive and negative values to each edge. Columns and rows in $W$ are sorted with TF first and PK second, so the order is $\text{TF}_1,\text{TF}_2,\text{TF}_3,\text{PK}_1,\text{PK}_2,\text{PK}_3$.

\begin{subequations}
\begin{align*}
\label{eq:detectable_example}
W &=
\begin{bmatrix} 
0 & 0 & 0 & 0 & 0 & -0.8 \\
0 & 0 & 0 & 0 & 0 & 0.7 \\
0 & 0 & 0 & 0 & -0.1 & 0.3 \\
1.1 & 0 & 0 & 0 & 0 & -0.4 \\
0 & 0 & 0 & 0 & 0 & 0 \\
0 & 0 & 0 & 0 & 0.5 & 0 \\
\end{bmatrix}
\\
|W|^\trans &=
\begin{bmatrix} 
0 & 0 & 0 & 1.1 & 0 & 0 \\
0 & 0 & 0 & 0 & 0 & 0 \\
0 & 0 & 0 & 0 & 0 & 0 \\
0 & 0 & 0 & 0 & 0 & 0 \\
0 & 0 & 0.1 & 0 & 0 & 0.5 \\
0.8 & 0.7 & 0.3 & 0.4 & 0 & 0 \\
\end{bmatrix}
\end{align*}
\end{subequations}

}


\only<3| handout:3>{

\begin{subequations}
\begin{align*}
I_T |W|^\trans \boldsymbol{1}
=
\sum_{k=0}^0 {|W|^\trans}^k
I_T |W|^\trans \boldsymbol{1}
&=
\begin{bmatrix} 
1.1 \\
0 \\
0 \\
0 \\
0 \\
0 \\
\end{bmatrix}
\\
\sum_{k=0}^1 {|W|^\trans}^k
I_T |W|^\trans \boldsymbol{1}
&=
\begin{bmatrix} 
1.1 \\
0 \\
0 \\
0 \\
0 \\
0.88 \\
\end{bmatrix}
\end{align*}
\end{subequations}

}

\only<4| handout:4>{

\begin{subequations}
\begin{align*}
\sum_{k=0}^2 {|W|^\trans}^k
I_T |W|^\trans \boldsymbol{1}
&=
\begin{bmatrix} 
1.1 \\
0 \\
0 \\
0 \\
0.44 \\
0.88 \\
\end{bmatrix}
\\
\sum_{k=0}^3 {|W|^\trans}^k
I_T |W|^\trans \boldsymbol{1}
&=
\begin{bmatrix} 
1.1 \\
0 \\
0 \\
0 \\
0.44 \\
0.88 \\
\end{bmatrix}
\end{align*}
\end{subequations}
}
\only<5| handout:5>{
% There is no change in the calculation from $K=2$ to $K=3$ so we set $K$ equal to 2 and get
\begin{equation*}
\label{eq:silent_example}
\boldsymbol{\omega} = \begin{bmatrix} 
1 \\
0 \\
0 \\
0 \\
1 \\
1 \\
\end{bmatrix}
\quad,\quad
M_s =
\begin{bmatrix} 
1 & 1 & 1 & 1 & 1 & 1 \\
1 & 1 & 1 & 0 & 0 & 0 \\
1 & 1 & 1 & 0 & 0 & 0 \\
1 & 1 & 1 & 0 & 0 & 0 \\
1 & 1 & 1 & 1 & 1 & 1 \\
1 & 1 & 1 & 1 & 1 & 1 \\
\end{bmatrix}
\end{equation*}
% This means that $\text{TF}_2,\text{TF}_3,\text{PK}_1$ are silent nodes. From $\boldsymbol{\omega}$ we can find the set of silent edges, which are all activity regulating edges onto any of the silent nodes. These entries in $W$ of silent edge is shown as zeros in $M_S$~(\autoref{eq:silent_example}), which is the masking matrix that indicates the edges to consider for a performance evaluation on an edge inference attempt. It is clear to spot the four edges in $W$ that are too be ignored if we were to infer edges for this network.
}
\stepcounter{equation}
\end{column}
\begin{column}{0.5\textwidth}
\begin{figure}[ht]
    \centering
    \includegraphics[width=0.6\textwidth]{theory/fig/unobservable.pdf}
    \caption{\textbf{Silent edges.} \textcolor{red}{red} = silent, \textcolor{black}{black} = detectable PK edge, dashed = transcription/translation, \textcolor{darkgray}{dark gray} = TF edge. }
    \label{fig:unobservable}
\end{figure}

\end{column}
\end{columns}
\end{frame}

\begin{frame}{Silent edges - removing $K$}
% As mentioned, $K$ in~\autoref{eq:detectable} is the length of the longest kinase cascade, or more precisely, the iteration where, if we computed further iterations of the sum, it would not change $\boldsymbol{\omega}$. We can simplify by letting $K$ tend to infinity.
for $K\rightarrow\infty$
\begin{equation}
\boldsymbol{\omega} =
\sign \sum_{k=0}^\infty {|W|^\trans}^k
I_T |W|^\trans \boldsymbol{1}
=
\sign \left( \left( \sum_{k=0}^\infty {|W|^\trans}^k \right)
I_T |W|^\trans \boldsymbol{1} \right)
\end{equation}
% If $I-|W|^\trans$ is invertible we can simplify the sum, using the same rule applied by Eberhardt~et~al.~(\autoref{eq:eber_converge.b}). We assume this holds since our system~(\autoref{eq:basic_eberhardt_extension}) is meant to reach equilibrium. It will not be assumed to hold for "gold standard" matrices discussed later~(\autoref{sec:dream_data}), since these does not hold edge values before preprocessing, and would diverge if used as such.
Simplify using same rule applied by Eberhardt~et~al.~(eq.~\ref{eq:eber_converge})
\begin{equation}
\label{eq:detectable_no_K}
\boldsymbol{\omega} =
\sign \left( \left( I - {|W|^\trans} \right)^{-1}
I_T |W|^\trans \boldsymbol{1} \right)
\end{equation}
% \autoref{eq:detectable_no_K} was tested on $W$ given in \autoref{eq:detectable_example}, which as expected produced $\boldsymbol{\omega}$ from \autoref{eq:silent_example} (a tolerance of $>\num{1e-14}$ was used instead of $\sign$ due to the imperfect nature of floating-point computation).
\end{frame}






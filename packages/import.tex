% loaded by beamer
% \usepackage[dvipsnames,table,xcdraw]{xcolor}
\usetheme[numbering=fraction]{metropolis}
\usecolortheme{beaver}
\setbeamercolor{background canvas}{bg=white}
% If you have more than one page of references, you want to tell beamer
% to put the continuation section label from the second slide onwards
\setbeamertemplate{frametitle continuation}[from second]
\setbeamertemplate{footline}{\begin{beamercolorbox}[wd=\textwidth,sep=3ex]{footline}\usebeamerfont{page number in head/foot}%
    \usebeamertemplate*{frame footer}
    \hfill%
    \textcolor{gray}{\insertsection}\space    
    \usebeamertemplate*{frame numbering}
  \end{beamercolorbox}%
}

% text encoding
\usepackage[utf8]{inputenc}
\usepackage[english]{babel}
\usepackage[T1]{fontenc}
\usepackage{lmodern}


% Greek letters in text
\usepackage{textgreek}

% quotations
\usepackage{fvextra} % to avoid warning in the next package
\usepackage{csquotes}


\usepackage{amsmath}
\usepackage{amsfonts}
\usepackage{amssymb}
\usepackage{physics}
\usepackage{siunitx}
\usepackage{gensymb}

\sisetup{per-mode=fraction}
% molar unit is added
\DeclareSIUnit\molar{M}

\DeclareMathOperator{\cov}{\text{cov}}
\DeclareMathOperator{\E}{\text{E}}
\DeclareMathOperator{\sign}{sgn}

% transpose version of T
\newcommand{\trans}{\intercal}

% environment to write descriptions of variables
\newenvironment{conditions}
  {\par\vspace{\abovedisplayskip}\noindent\begin{tabular}{>{$}l<{$} @{${}={}$} l}}
  {\end{tabular}\par\vspace{\belowdisplayskip}}


\usepackage{float} % better positioning of images
\usepackage{graphicx}
\usepackage[font={small,it}]{caption}
\usepackage{subcaption}
% to wrap text around figure so it can be displayed next to text
\usepackage{wrapfig}
\usepackage{sidecap}


% add figure list
\graphicspath{{./fig/}}
\usepackage{listings}


\usepackage{tabularx}
% define Y column type that is X with centering
\newcolumntype{Y}{>{\centering\arraybackslash}X}
% creates a type R to specify right adjustment instead of left
\newcolumntype{R}{>{\raggedleft\arraybackslash}X}
\usepackage{wrapfig}
\usepackage{booktabs}
% make it possible to make tables span more than one page
\usepackage{ltxtable}
% add boxes functionality to force line break in table cell
\usepackage{pbox}

% make a good looking tilde
\newcommand{\realtilde}{\raisebox{0.5ex}{\texttildelow}}


% appendix
% \usepackage[toc,page]{appendix}
% appendix is not defined implicitly
% \newcommand*{\Appendixautorefname}{Appendix}


% makes sure figures and tables stay in their section
\usepackage[section]{placeins}

% justification is raggedright (i.e. left aligned)
% singlelinecheck=off means that the justification setting is used even when the caption is only a single line long. 
% if singlelinecheck=on, then caption is always centered when the caption is only one line.
\captionsetup[subfigure]{singlelinecheck=off,justification=raggedright}

% % if you want to add numbering on parapraphs
% \usepackage{titlesec}
% % \setcounter{secnumdepth}{4}
% \titleformat{\paragraph}
% {\normalfont\normalsize\bfseries}{\theparagraph}{1em}{}
% \titlespacing*{\paragraph}
% {0pt}{3.25ex plus 1ex minus .2ex}{1.5ex plus .2ex}

\usepackage{array} % don't know what is used for

% nomenclature
% \usepackage{nomencl}

% bibliography. Maybe use bibtex instead of biber as backend
\usepackage[backend=biber,sorting=none,style=numeric]{biblatex}
\AtEveryBibitem{%
  \clearfield{issn} % Remove issn
  \clearfield{isbn} % Remove issn
  \clearfield{doi} % Remove doi

  \ifentrytype{misc}{}{% Remove url except for @misc
    \clearfield{url}
  }
}
\addbibresource{references.bib}
% import after adding bibliography in main in order to unescape _ and ~ in urls
\DeclareSourcemap{
\maps{
% Replaces '{\_}', '{_}' or '\_' with just '_'
    \map{
        \step[fieldsource=url,
            match=\regexp{\{\\\_\}|\{\_\}|\\\_},replace=\regexp{\_}]
        }
    \map{
        \step[fieldsource=doi,
            match=\regexp{\{\\\_\}|\{\_\}|\\\_},replace=\regexp{\_}]
        }
% Replaces '{'$\sim$'}', '$\sim$' or '{~}' with just '~'
    \map{ 
        \step[fieldsource=url,
            match=\regexp{\{\$\\sim\$\}|\{\~\}|\$\\sim\$},replace=\regexp{\~}]
        }
    \map{ 
        \step[fieldsource=doi,
            match=\regexp{\{\$\\sim\$\}|\{\~\}|\$\\sim\$},replace=\regexp{\~}]
        }
    }
}

% make bibliography entries smaller
\renewcommand\bibfont{\tiny}
% replace weird icon with numbering
\setbeamertemplate{bibliography item}{\insertbiblabel}


% header
\usepackage{fancyhdr}
\setlength{\headheight}{15.2pt}
% paragraphs of text start with no indent
\setlength{\parindent}{0pt}
% paragraphs of text start with a little skip from last text
\setlength{\parskip}{1ex plus 0.5ex minus 0.2ex}

% this package gives us minted to show code and Verbatim to set verbatim font size
\usepackage{minted}
\usepackage{parskip}
\definecolor{LightGray}{gray}{0.97}
\setminted[python]{baselinestretch=1.2,tabsize=4,bgcolor=LightGray,fontsize=\footnotesize,linenos}
\setminted[bash]{baselinestretch=1.2,bgcolor=LightGray,fontsize=\footnotesize,linenos}


% hyperref uses xcolor 
\usepackage{hyperref}
\hypersetup{
  linkcolor  = violet,
  citecolor  = YellowOrange,
  urlcolor   = Aquamarine,
  colorlinks = true,
}
% alt to hyperref's autoref
\usepackage{cleveref}

% insert image in title page or other difficult place
\usepackage{tikz}
\usetikzlibrary{positioning}



Signal transduction
\begin{itemize}
    \item Phosphorylation
% Signal transduction inside the cell is mostly performed through the addition or removal of phosphates as phosphorylation or dephosphorylation of proteins, although there are many other types of signal transduction and protein regulation such as G-protein coupled receptors mediating trans-membrane signals, second messengers such as calcium ions and lipid messengers. 
    \item TF, $\text{KP} = \text{PK} \cup \text{PP}$, V (any genes)
% For intracellular signals leading to gene regulation, gene regulatory proteins can broadly be categorizes as either transcription factors, kinases or phosphatases, where the TFs have direct effects on the gene transcription and kinases and phosphatases regulates the activity of TFs as well as each other. TFs and their regulons, which are the set of genes regulated by a TF, are largely known, while kinase regulation is not as well documented.

% There are about a factor of 10 less phosphatases of yeast than kinases. They have a much smaller specificity in regulation targets and can, to some extent, be seen as serving a general clean-up role by removing phosphates from regulated proteins in order to let cellular signals decay.
% Phosphorylation can generally be seen as a on/off switch for a specific site on the protein, with one state of phosphorylation inducing an active conformation while the other stabilizes a more inactive conformation. It is atypical that a transcription factor will regulate one set of genes while phosphorylated and another while dephosphorylated. It can both be a phosphorylation that activates a proteins function or deactivates it, however since kinases have more specificity, kinase chains will often send a cellular signal when phosphorylated and have the signal decay with dephosphorylation mediated by phosphatases. Proteins are often modeled as either being on or off, but multiple post-translational modifications can have combined effects leading to multiple levels of protein activity for a given protein.
\end{itemize}

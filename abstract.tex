\section*{Abstract}

Fundamental to understanding life is understanding the biological cell, where gene regulation dictates function and behavior. The cell and its regulation is an intricate system of interacting constituents. Large quantities of data has been collected experimentally. Researchers have increasingly been applying systems biology approached by studying the system as a whole, rather than as single reactions, due to complexities of the vast network of interacting components, and due to imperfections of data. Such approaches rely on models, where assumptions and choices in simplifications are key. Gene regulation is often modeled as a system of transcription factors and genes. Here, we attempt to include protein regulation and the cyclic nature arising from the participation of gene products in future iterations of gene regulation. Biologically inspired graph models are presented and studied, both for simulation purposes, as well as for protein-protein and protein-DNA interaction inference. The simulation is shown to produce intuitively meaningful data, and the main inference model presented is shown to perform well on data simulated with biologically meaningful non-linear gene regulation. An inference attempt on experimental data indicated that there is more complexity to take into account. The main method was compared to previously published work, of which is it is an extension, which it was found to significantly outperform. Examples of other models and suggestions for further developments are discussed.



